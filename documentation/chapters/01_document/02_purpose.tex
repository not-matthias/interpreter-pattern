\chapter{Purpose}

\section{When should it be used?}
\label{sec:when-to-use}

You should not use the interpreter pattern for every problem. It has, like the other design patterns, its use cases. However, you should always use it when:  
\begin{itemize}
    \item there's a language to interpret and you can represent the statements as an AST.

    \item the grammar is simple.
    \begin{itemize}
        \item In large projects, the class hierarchy becomes large and unmaintainable. 
        \item Parser are a great alternative in such cases.
        \item They don't use an AST, which can save space and possibly time.
    \end{itemize}

    \item efficiency is not a critical concern.
    \begin{itemize}
        \item It's usually more efficient, when translating the parse tree to another form.
        \item For example, regular expressions are often transformed into state machines. 
        \item The interpreter pattern could there still be used to translate it. 
    \end{itemize}
\end{itemize}

% Source: https://github.com/iluwatar/java-design-patterns/tree/master/interpreter